%!TEX enableShellEscape = yes
% (The above line makes atom's latex package compile with -shell-escape
% for minted, and is just ignored by other systems.)
\documentclass{article}

\usepackage{fullpage}
\usepackage{color}
\usepackage{amsmath,amssymb}
\usepackage{url}
\usepackage{verbatim}
\usepackage{graphicx}
\usepackage{parskip}
\usepackage{amssymb}
\usepackage{hyperref}
\author{Nan Wang\\30566426}
% Use one or the other of these for displaying code.w
% NOTE: If you get
%  ! Package minted Error: You must invoke LaTeX with the -shell-escape flag.
% and don't want to use minted, just comment out the next line
\usepackage{minted} 
\BeforeBeginEnvironment{minted}{\begingroup\color{black}} 
\AfterEndEnvironment{minted}{\endgroup} 
\setminted{autogobble,breaklines,breakanywhere,linenos}

\usepackage{listings}

% Colours
\definecolor{blu}{rgb}{0,0,1}
\newcommand{\blu}[1]{{\textcolor{blu}{#1}}}
\definecolor{gre}{rgb}{0,.5,0}
\newcommand{\gre}[1]{\textcolor{gre}{#1}}
\definecolor{red}{rgb}{1,0,0}
\newcommand{\red}[1]{\textcolor{red}{#1}}
\definecolor{pointscolour}{rgb}{0.6,0.3,0}

% answer commands
\newcommand\ans[1]{\par\gre{Answer: #1}}
\newenvironment{answer}{\par\begingroup\color{gre}Answer: }{\endgroup}
\let\ask\blu
\let\update\red
\newenvironment{asking}{\begingroup\color{blu}}{\endgroup}
\newcommand\pts[1]{\textcolor{pointscolour}{[#1~points]}}

% Math
\def\R{\mathbb{R}}
\DeclareMathOperator*{\argmax}{arg\,max}
\DeclareMathOperator*{\argmin}{arg\,min}
\newcommand{\norm}[1]{\lVert #1 \rVert}
\newcommand{\mat}[1]{\begin{bmatrix}#1\end{bmatrix}}

% LaTeX
\newcommand{\centerfig}[2]{\begin{center}\includegraphics[width=#1\textwidth]{#2}\end{center}}


\begin{document}

  \title{CPSC 340 Assignment 1 (due 2022-Jan-19 at 11:59 pm)}

  \date{}
  \maketitle



  \textbf{Commentary on Assignment 1}: CPSC 340 is tough because it combines knowledge and skills across several disciplines. To succeed
  in the course, you will need to know or very quickly get up to speed on:
  \begin{itemize}
  \item Basic Python programming, including NumPy and plotting with matplotlib.
  \item Math to the level of the course prerequisites: linear algebra, multivariable calculus, some probability.
  \item Statistics, algorithms and data structures to the level of the course prerequisites.
  \item Some basic LaTeX skills so that you can typeset equations and submit your assignments.
  \end{itemize}

  This assignment will help you assess whether you are prepared for this course. We anticipate that each
  of you will have different strengths and weaknesses, so don't be worried if you struggle with \emph{some} aspects
  of the assignment. \textbf{But if you find this assignment
  to be very difficult overall, that is a warning sign that you may not be prepared to take CPSC 340
  at this time.} Future assignments will be more difficult than this one (and probably around the same length).

  Questions 1-4 are on review material, that we expect you to know coming into the course. The rest is new CPSC 340 material from the first few lectures.

  \textbf{A note on the provided code:} in the \texttt{code} directory we provide you with a file called
  \texttt{main.py}. This file, when run with different arguments, runs the code for different
  parts of the assignment. For example,
  \begin{verbatim}
  python main.py 6.2
  \end{verbatim}
  runs the code for Question 6.2. At present, this should do nothing (throws a \texttt{NotImplementedError}), because the code
  for Question 6.2 still needs to be written (by you). But we do provide some of the bits
  and pieces to save you time, so that you can focus on the machine learning aspects.
  For example, you'll see that the provided code already loads the datasets for you.
  The file \texttt{utils.py} contains some helper functions.
  You don't need to read or modify the code in there.
  To complete your assignment, you will need to modify \texttt{grads.py}, \texttt{main.py}, \texttt{decision\string_stump.py} and \texttt{simple\string_decision.py} (which you'll need to create).


  \section*{Instructions \pts{5}}

  The above points are allocated for following the submission instructions: \\\url{https://github.com/UBC-CS/cpsc340-2021w1/blob/main/docs/submissionInstructions.pdf}

  \vspace{1em}
  We use \ask{blue} to highlight the deliverables that you must answer/do/submit with the assignment.

  \section{Linear Algebra Review \pts{17}}

  For these questions you may find it helpful to review these notes on linear algebra:\\
  \url{http://www.cs.ubc.ca/~schmidtm/Documents/2009_Notes_LinearAlgebra.pdf}

  \subsection{Basic Operations \pts{7}}

  Use the definitions below,
  \[
  \alpha = 2,\quad
  x = \left[\begin{array}{c}
  0\\
  1\\
  2\\
  \end{array}\right], \quad
  y = \left[\begin{array}{c}
  3\\
  4\\
  5\\
  \end{array}\right],\quad
  z = \left[\begin{array}{c}
  1\\
  4\\
  -2\\
  \end{array}\right],\quad
  A = \left[\begin{array}{ccc}
  3 & 2 & 2\\
  1 & 3 & 1\\
  1 & 1 & 3
  \end{array}\right],
  \]
  and use $x_i$ to denote element $i$ of vector $x$.
  \ask{Evaluate the following expressions} (you do not need to show your work).
  \begin{enumerate}
  \item $\sum_{i=1}^n x_iy_i$ (inner product).
  \begin{answer}
    14
  \end{answer}
  \item $\sum_{i=1}^n x_i z_i$ (inner product between orthogonal vectors).
  \begin{answer}
  0
  \end{answer}
  \item $\alpha(x+z)$ (vector addition and scalar multiplication)
  \begin{answer}
  $\left[\begin{array}{c}
       2 \\
       10\\
       0\\
  \end{array}\right]$
  \end{answer} 
  
  \item $x^Tz + \norm{x}$ (inner product in matrix notation and Euclidean norm of $x$).
  \begin{answer}
 $ \sqrt{5}$
  \end{answer}
  \item $Ax$ (matrix-vector multiplication).
  \begin{answer}
   $\left[\begin{array}{c}
       6  \\
       5 \\
       7
  \end{array} \right]$
  \end{answer}
  \item $x^TAx$ (quadratic form).
  \begin{answer}
   $19$
  \end{answer}
  \item $A^TA$ (matrix tranpose and matrix multiplication).
  \begin{answer}
   $\left[ \begin{array}{ccc}
        11 & 10 & 10 \\
        10 & 14 & 10 \\
        10 & 10 & 14 
   \end{array}\right]$
  \end{answer}
  \end{enumerate}

  \subsection{Matrix Algebra Rules \pts{10}}

  Assume that $\{x,y,z\}$ are $n \times 1$ column vectors, $\{A,B,C\}$ are $n \times n$ real-valued matrices, $0$ is the zero matrix of appropriate size, and $I$ is the identity matrix of appropriate size. \ask{State whether each of the below is true in general} (you do not need to show your work).

  \begin{enumerate}
  \item $x^Ty = \sum_{i=1}^n x_iy_i$.
  %
  \ans{True}
  \item $x^Tx = \norm{x}^2$.
  %
  \ans{True}
  \item $x^Tx = xx^T$.
  %
  \ans{False}
  \item $(x-y)^T(x-y) = \norm{x}^2 - 2x^Ty + \norm{y}^2$.
  %
  \ans{True}
  \item $AB=BA$.
  %
  \ans{False}
  \item $A^T(B + IC) = A^TB + A^TC$.
  %
  \ans{True}
  \item $(A + BC)^T = A^T + B^TC^T$.
  %
  \ans{False}
  \item $x^TAy = y^TA^Tx$.
  %
  \ans{False}
  \item $A^TA = AA^T$ if $A$ is a symmetric matrix.
  %
  \ans{True}
  \item $A^TA = 0$ if the columns of $A$ are orthonormal.
  %
  \ans{False}
  \end{enumerate}


  \clearpage 
  \section{Probability Review \pts{16}}


  For these questions you may find it helpful to review these notes on probability:\\
  \url{http://www.cs.ubc.ca/~schmidtm/Courses/Notes/probability.pdf}\\
  And here are some slides giving visual representations of the ideas as well as some simple examples:\\
  \url{http://www.cs.ubc.ca/~schmidtm/Courses/Notes/probabilitySlides.pdf}

  \subsection{Rules of probability \pts{6}}

  \ask{Answer the following questions.} You do not need to show your work.

  % Combinations/enumeration questions, independence question

  \begin{enumerate}
  \item You are offered the opportunity to play the following game: your opponent rolls 2 regular 6-sided dice. If the difference between the two rolls is at least 3, you win \$30. Otherwise, you get nothing. What is a fair price for a ticket to play this game once? In other words, what is the expected value of playing the game?
  %
  \begin{answer}
    10
  \end{answer}
  \item Consider two events $A$ and $B$ such that $\Pr(A \cap B)=0$ (they are mutually exclusive). If $\Pr(A) = 0.4$ and $\Pr(A \cup B) = 0.95$, what is $\Pr(B)$? Note: $\Pr(A \cap B)$ means
  ``probability of $A$ and $B$'' while $p(A \cup B)$ means ``probability of $A$ or $B$''. It may be helpful to draw a Venn diagram.
  %
  \begin{answer}
    0.55
  \end{answer}
  \item Instead of assuming that $A$ and $B$ are mutually exclusive ($\Pr(A \cap B) = 0)$, what is the answer to the previous question if we assume that $A$ and $B$ are independent?
  %
  \begin{answer}
   $\frac{11}{12}$
  \end{answer}

  \end{enumerate}

  \subsection{Bayes Rule and Conditional Probability \pts{10}}

  \ask{Answer the following questions.} You do not need to show your work.

  Suppose a drug test produces a positive result with probability $0.97$ for drug users, $P(T=1 \mid D=1)=0.97$. It also produces a negative result with probability $0.99$ for non-drug users, $P(T=0 \mid D=0)=0.99$. The probability that a random person uses the drug is $0.0001$, so $P(D=1)=0.0001$.

  \begin{enumerate}
  \item What is the probability that a random person would test positive, $P(T=1)$?
  \begin{answer}
    0.0100087
  \end{answer}
  \item In the above, do most of these positive tests come from true positives or from false positives?
  %
  \begin{answer}
    Flase positive
  \end{answer}
  \item What is the probability that a random person who tests positive is a user, $P(D=1 \mid T=1)$?
  \begin{answer}
    0.00096916
  \end{answer}
  \item Suppose you have given this test to a random person and it came back positive, are they likely to be a drug user?
  %
  \begin{answer}
    No
  \end{answer}
  \item Suppose you are the designer of this drug test. You may change how the test is conducted, which may influence factors like false positive rate, false negative rate, and the number of samples collected. What is one factor you could change to make this a more useful test?
  %
  \begin{answer}
    False positive rate
  \end{answer}
  \end{enumerate}


  \clearpage
  \section{Calculus Review \pts{23}}



  \subsection{One-variable derivatives \pts{8}}
  \label{sub.one.var}

  \ask{Answer the following questions.} You do not need to show your work.

  \begin{enumerate}
  \item Find the derivative of the function $f(x) = 3x^3 -2x + 5$.
  %
  \begin{answer}
    $9x^2 - 2$
  \end{answer}
  \item Find the derivative of the function $f(x) = x(1-x)$.
  %
  \begin{answer}
    $1-2x$
  \end{answer}
  \item Let $p(x) = \frac{1}{1+\exp(-x)}$ for $x \in \R$. Compute the derivative of the function $f(x) = x-\log(p(x))$ and simplify it by using the function $p(x)$.
  \begin{answer}
    $p(x)$
  \end{answer}
  \end{enumerate}
  Remember that in this course we will $\log(x)$ to mean the ``natural'' logarithm of $x$, so that $\log(\exp(1)) = 1$. Also, observe that $p(x) = 1-p(-x)$ for the final part.

  \subsection{Multi-variable derivatives \pts{5}}
  \label{sub.multi.var}

  \ask{Compute the gradient vector $\nabla f(x)$ of each of the following functions.} You do not need to show your work.
  \begin{enumerate}
  \item $f(x) = x_1^2 + \exp(x_1 + 3x_2)$ where $x \in \R^2$.
  %
  \begin{answer}
    $\left[ \begin{array}{c}
      2x_1 + exp(x_1 + 3x_2) \\
      3exp(x_1 + 3x_2) \\
    \end{array} \right]$
  \end{answer}
  \item $f(x) = \log\left(\sum_{i=1}^3\exp(x_i)\right)$ where $x \in \R^3$ (simplify the gradient by defining $Z = \sum_{i=1}^3\exp(x_i)$).
  %
  \begin{answer}
    $ \left[ \begin{array}{c}
      exp(x_1)\\
      exp(x_2)\\
      exp(x_3)\\
    \end{array}\right]\frac{1}{Z} $
  \end{answer}
  \item $f(x) = a^Tx + b$ where $x \in \R^3$ and $a \in \R^3$ and $b \in \R$.
  %
  \begin{answer}
    $a^T$
  \end{answer}
  \item $f(x) = \frac12 x^\top A x$ where $A=\left[ \begin{array}{cc}
  2 & -1 \\
  -1 & 2 \end{array} \right]$ and $x \in \mathbb{R}^2$.
  %
  \begin{answer}
    $ \left[ \begin{array}{c}
      2x_1 - x_2 \\
      2x_2 - x_1 \\
    \end{array} \right] $
  \end{answer}
  \item $f(x) = \frac{1}{2}\norm{x}^2$ where $x \in \R^d$.
  %
  \begin{answer}
    $ x$
  \end{answer}
  \end{enumerate}

  Hint: it may be helpful to write out the linear algebra expressions in terms of summations.


  \subsection{Optimization \pts{6}}

  \ask{Find the following quantities.} You do not need to show your work.
  %You can/should use your results from parts \ref{sub.one.var} and \ref{sub.multi.var} as part of the process.

  \begin{enumerate}
  \item $\min \, 3x^2-2x+5$, or, in words, the minimum value of the function $f(x) = 3x^2 -2x + 5$ for $x \in \R$.
  %
  \begin{answer}
    $\frac{14}{3}$
  \end{answer}
  \item $\max_{x \in [0, 1]} x(1-x)$
  %
  \begin{answer}
    0.25
  \end{answer}
  \item $\min_{x \in [0, 1]} \, x(1-x)$
  %
  \begin{answer}
    0
  \end{answer}
  \item $\argmax_{x \in [0,1]} \, x(1-x)$
  %
  \begin{answer}
    0.5
  \end{answer}
  \item $\min_{x \in [0, 1]^2} \, x_1^2 + \exp(x_2)$ -- in other words $x_1\in [0,1]$ and $x_2\in [0,1]$
  %
  \begin{answer}
    1
  \end{answer}
  \item $\argmin_{x \in [0,1]^2} \, x_1^2 + \exp(x_2)$ where $x \in [0,1]^2$.
  %
  \begin{answer}
    $ \left[ \begin{array}{c}
      0\\
      0\\
    \end{array} \right]$
  \end{answer}
  \end{enumerate}

  Note: the notation $x\in [0,1]$ means ``$x$ is in the interval $[0,1]$'', or, also equivalently, $0 \leq x \leq 1$.

  Note: the notation ``$\max f(x)$'' means ``the value of $f(x)$ where $f(x)$ is maximized'', whereas ``$\argmax  f(x)$'' means ``the value of $x$ such that $f(x)$ is maximized''.
  Likewise for $\min$ and $\argmin$. For example, the min of the function $f(x)=(x-1)^2$ is $0$ because the smallest possible value is $f(x)=0$,
  whereas the arg min is $1$ because this smallest value occurs at $x=1$. The min is always a scalar but the $\argmin$ is a value of $x$, so it's a vector
  if $x$ is vector-valued.

  \subsection{Derivatives of code \pts{4}}

  Your repository contains a file named \texttt{grads.py} which defines several Python functions that take in an input variable $x$, which we assume to be a 1-d array (in math terms, a vector).
  It also includes (blank) functions that return the corresponding gradients.
  For each function, \ask{write code that computes the gradient of the function} in Python.
  You should do this directly in \texttt{grads.py}; no need to make a fresh copy of the file. When finished, you can run \texttt{python main.py 3.4} to test out your code. \ask{Include this code following the instructions in the submission instructions.}

  \begin{minted}{python}
def foo_grad(x):
    return 6* x**5
  \end{minted}
  \begin{minted}{python}
def bar_grad(x):
# If there is no zero element, then report the normal gradient 
    if np.prod(x) != 0: 
        return np.prod(x) * x**-1
# If there are more than two zeros, the gradient will be a zero vector
    else:
        if x.size - np.count_nonzero(x) >= 2:
            return np.zeros(x.size)
#If there is only one zero, the derivative of the zero element will be nonzero, and all other elements will be zero
        else:
            prod = 1
            for x_i in x:
                if x_i != 0:
                    prod *= x_i
            result = np.where(x == 0, prod, 0)
            return result
  \end{minted}

  Hint: it's probably easiest to first understand on paper what the code is doing, then compute
  the gradient, and then translate this gradient back into code.

  Note: do not worry about the distinction between row vectors and column vectors here.
  For example, if the correct answer is a vector of length 5, we'll accept numpy arrays
  of shape \texttt{(5,)} (a 1-d array) or \texttt{(5,1)} (a column vector) or
  \texttt{(1,5)} (a row vector). In future assignments we will start to be more careful
  about this.

  Warning: Python uses whitespace instead of curly braces to delimit blocks of code.
  Some people use tabs and other people use spaces. My text editor (Atom) inserts 4 spaces (rather than tabs) when
  I press the Tab key, so the file \texttt{grads.py} is indented in this manner (and indeed, this is standard Python style that you should probably also follow). If your text editor inserts tabs,
  Python will complain and you might get mysterious errors... this is one of the most annoying aspects
  of Python, especially when starting out. So, please be aware of this issue! And if in doubt you can just manually
  indent with 4 spaces, or convert everything to tabs. For more information
  see \url{https://www.youtube.com/watch?v=SsoOG6ZeyUI}.


  \clearpage
  \section{Algorithms and Data Structures Review \pts{11}}

  \subsection{Trees \pts{2}}

  \ask{Answer the following questions.} You do not need to show your work. We'll define ``depth'' as the maximum number of edges you need to traverse to get from the root of the tree to a leaf of the tree. In other words, if you're thinking about nodes, then the leaves are not included in the depth, so a complete tree with depth $1$ has 3 nodes with 2 leaves.


  \begin{enumerate}
  \item What is the minimum depth of a binary tree with 64 leaf nodes?
  %
  \begin{answer}
    6
  \end{answer}
  \item What is the minimum depth of binary tree with 64 nodes (including leaves and all other nodes)?
  %
  \begin{answer}
    6
  \end{answer}
  \end{enumerate}

  \subsection{Common Runtimes \pts{5}}

  \ask{Answer the following questions using big-$O$ notation} You do not need to show your work.
  Here, the word ``list'' means e.g.\ a Python \texttt{list} -- an array, not a linked list.
  \begin{enumerate}
  \item What is the cost of finding the largest number in an unsorted list of $n$ numbers?
  %
  \begin{answer}
    O(n)
  \end{answer}
  \item What is the cost of finding the smallest element greater than 0 in a \emph{sorted} list with $n$ numbers.
  %
  \begin{answer}
    O(log n)
  \end{answer}
  \item What is the cost of finding the value associated with a key in a hash table with $n$ numbers? \\(Assume the values and keys are both scalars.)
  %
  \begin{answer}
    O(1)
  \end{answer}
  \item What is the cost of computing the inner product $a^Tx$, where $a$ is $d \times 1$ and $x$ is $d \times 1$?
  %
  \begin{answer}
    O(d)
  \end{answer}
  \item What is the cost of computing the quadratic form $x^TAx$ when $A$ is $d \times d$ and $x$ is $d \times 1$?
  %
  \begin{answer}
    $O(d^2)$
  \end{answer}
  \end{enumerate}

  \subsection{Running times of code \pts{4}}

  Your repository contains a file named \texttt{bigO.py}, which defines several functions
  that take an integer argument $N$. For each function, \ask{state the running time as a function of $N$, using big-O notation}.
  %
  \begin{answer}
    func1: O(N); 
    func2: O(N); 
    func3: O(1); 
    func4: O($N^2$)
  \end{answer}


  \clearpage
  \section{Data Exploration \pts{5}}


  Your repository contains the file \texttt{fluTrends.csv}, which contains estimates
  of the influenza-like illness percentage over 52 weeks on 2005-06 by Google Flu Trends.
  Your \texttt{main.py} loads this data for you and stores it in a pandas DataFrame \texttt{X},
  where each row corresponds to a week and each column
  corresponds to a different
  region. If desired, you can convert from a DataFrame to a raw numpy array with \texttt{X.values()}.

  \subsection{Summary Statistics \pts{2}}

  \ask{Report the following statistics}:
  \begin{enumerate}
  \item The minimum, maximum, mean, median, and mode of all values across the dataset.
  \begin{answer}
    Min: 0.352; Max: 4.862; Mean: 1.325; Median: 1.159; Mode: 0.770
  \end{answer}
  \item The $5\%$, $25\%$, $50\%$, $75\%$, and $95\%$ quantiles of all values across the dataset.
  \begin{answer}
    5 quantile: 0.465; 25 quantile: 0.718; 50 quantile: 1.159; 75 quantile: 1.813; 95 quantile: 2.624
  \end{answer}

  \item The names of the regions with the highest and lowest means, and the highest and lowest variances.% \item The pairs of regions with the highest and lowest correlations.
  \begin{answer}
    Highest mean: WtdILI; Lowest mean: Pac; Highest variance: Mtn; Lowest variance: Pac 
  \end{answer}
  \end{enumerate}
  In light of the above, \ask{is the mode a reliable estimate of the most ``common" value? Describe another way we could give a meaningful ``mode" measurement for this (continuous) data.} Note: the function \texttt{utils.mode()} will compute the mode value of an array for you.

  % Note: by ``lowest'' we mean most negative, not the closest to zero.

  


  \subsection{Data Visualization \pts{3}}

  Consider the figure below.

  \centerfig{.9}{./figs/visualize-unlabeled}

  The figure contains the following plots, in a shuffled order:
  \begin{enumerate}
  \item A single histogram showing the distribution of \emph{each} column in $X$.
  \begin{answer}
    Plot D. It's a histogram containing the information of each column
  \end{answer}
  \item A histogram showing the distribution of each the values in the matrix $X$.
  \begin{answer}
    Plot C. It's a histogram of all the values in the dataset
  \end{answer}
  \item A boxplot grouping data by weeks, showing the distribution across regions for each week.
  \begin{answer}
    Plot B. It's boxplot in weeks.
  \end{answer}
  \item A plot showing the illness percentages over time.
  \begin{answer}
    Plot A. It's a line graph of illness percentage over time.
  \end{answer}
  \item A scatterplot between the two regions with highest correlation.
  \begin{answer}
    Plot F. The saccterplot looks pretty much like the 45 degree line, indicating the two variables are very similar.
  \end{answer}
  \item A scatterplot between the two regions with lowest correlation.
  \begin{answer}
    Plot E. In the scatterplot, there are many outliers.
  \end{answer}
  \end{enumerate}

  \ask{Match the plots (labeled A-F) with the descriptions above (labeled 1-6), with an extremely brief (a few words is fine) explanation for each decision.}

  




  \clearpage
  \section{Decision Trees \pts{23}}

  If you run \texttt{python main.py 6}, it will load a dataset containing longitude
  and latitude data for 400 cities in the US, along with a class label indicating
  whether they were a ``red" state or a ``blue" state in the 2012
  election.\footnote{The cities data was sampled from \url{http://simplemaps.com/static/demos/resources/us-cities/cities.csv}. The election information was collected from Wikipedia.}
  Specifically, the first column of the variable $X$ contains the
  longitude and the second variable contains the latitude,
  while the variable $y$ is set to $0$ for blue states and $1$ for red states.
  After it loads the data, it plots the data and then fits two simple
  classifiers: a classifier that always predicts the
  most common label ($0$ in this case) and a decision stump
  that discretizes the features (by rounding to the nearest integer)
  and then finds the best equality-based rule (i.e., check
  if a feature is equal to some value).
  It reports the training error with these two classifiers, then plots the decision areas made by the decision stump.
  The plot is shown below:

  \centerfig{0.7}{./figs/q6_decisionBoundary}

  As you can see, it is just checking whether the latitude equals 35 and, if so, predicting red (Republican).
  This is not a very good classifier.

  \subsection{Splitting rule \pts{1}}

  Is there a particular type of features for which it makes sense to use an equality-based splitting rule rather than the threshold-based splits we discussed in class?
  \begin{answer}
    Categorical features
  \end{answer}


  \subsection{Decision Stump Implementation \pts{8}}\label{sec6.2}

  The file \texttt{decision\string_stump.py} contains the class \texttt{DecisionStumpEquality} which
  finds the best decision stump using the equality rule and then makes predictions using that
  rule. Instead of discretizing the data and using a rule based on testing an equality for
  a single feature, we want to check whether a feature is above or below a threshold and
  split the data accordingly (this is a more sane approach, which we discussed in class).
  \ask{Create a \texttt{DecisionStumpErrorRate} class to do this, and report the updated error you
  obtain by using inequalities instead of discretizing and testing equality.
  Submit your class definition code as a screenshot or using the \texttt{lstlisting} environment.
  Also submit the generated figure of the classification boundary.}

  Hint: you may want to start by copy/pasting the contents \texttt{DecisionStumpEquality} and then make modifications from there. %You can make modifications from there. % But you are asked to keep them separate rather than directly modifying the provided code so that you can keep both versions in tact and compare them.
  % Note: please keep the same variable names, as subsequent parts of this assignment rely on this!

  \centerfig{0.7}{./figs/q6_2_decisionBoundary}
  \begin{answer}
    Decision Stump with inequality rule error: 0.253. See code \hyperref[code6.2]{here}.
  \end{answer}
  



  \subsection{Decision Stump Info Gain Implementation \pts{8}}\label{sec6.3}

  In \texttt{decision\string_stump.py}, \ask{create a \texttt{DecisionStumpInfoGain} class that
  fits using the information gain criterion discussed in lecture.
  Report the updated error you obtain.
  Submit your class definition code as a screenshot or using the \texttt{lstlisting} environment.
  Submit the classification boundary figure.}

  Notice how the error rate changed. Are you surprised? If so, hang on until the end of this question!

  Note: even though this data set only has 2 classes (red and blue), your implementation should work
  for any number of classes, just like \texttt{DecisionStumpEquality} and \texttt{DecisionStumpErrorRate}.

  Hint: take a look at the documentation for \texttt{np.bincount}, at \\
  \url{https://docs.scipy.org/doc/numpy/reference/generated/numpy.bincount.html}.
  The \texttt{minlength} argument comes in handy here to deal with a tricky corner case:
  when you consider a split, you might not have any cases of a certain class, like class 1,
  going to one side of the split. Thus, when you call \texttt{np.bincount}, you'll get
  a shorter array by default, which is not what you want. Setting \texttt{minlength} to the
  number of classes solves this problem.

  \centerfig{0.7}{./figs/q6_3_decisionBoundary}
  \begin{answer}
    Decision Stump with info gain rule error: 0.325. See code \hyperref[code6.3]{here}.
  \end{answer}
  
  

  \subsection{Hard-coded Decision Trees \pts{2}}

  Once your \texttt{DecisionStumpInfoGain} class is finished, running \texttt{python main.py 6.4} will fit
  a decision tree of depth~2 to the same dataset (which results in a lower training error).
  Look at how the decision tree is stored and how the (recursive) \texttt{predict} function works.
  \ask{Using the splits from the fitted depth-2 decision tree, write a hard-coded version of the \texttt{predict}
  function that classifies one example using simple if/else statements
  (see the Decision Trees lecture). Submit this code as a plain text, as a screenshot or using the \texttt{lstlisting} environment.}

  Note: this code should implement the specific, fixed decision tree
  which was learned after calling \texttt{fit} on this particular data set. It does not need to be a learnable model.
  You should just hard-code the split values directly into the code.
  Only the \texttt{predict} function is needed.

  Hint: if you plot the decision boundary you can do a quick visual check to see if your code is consistent with the plot.

  \begin{minted}{python}
  def predict(X):
    if X[0] <= -80.305106:
            y = 0
    else:
        if X[1] <= 37.669007:
            y = 0
        else:
            y = 1
    return y
   \end{minted}



  \subsection{Decision Tree Training Error \pts{2}}

  Running \texttt{python main.py 6.5} fits decision trees of different depths using the following different implementations:
  \begin{enumerate}
  \item A decision tree using \texttt{DecisionStumpErrorRate}
  \item A decision tree using \texttt{DecisionStumpInfoGain}
  \item The \texttt{DecisionTreeClassifier} from the popular Python ML library \emph{scikit-learn}
  \end{enumerate}

  % (3) and (3) above use a more sophisticated splitting criterion called the information gain, instead of just the classification accuracy.
  Run the code and look at the figure.
  \ask{Describe what you observe. Can you explain the results?} Why is approach (1) so disappointing? Also, \ask{submit a classification boundary plot of the model with the lowest training error}.

  Note: we set the \verb|random_state| because sklearn's \texttt{DecisionTreeClassifier} is non-deterministic. This is probably
  because it breaks ties randomly.

  Note: the code also prints out the amount of time spent. You'll notice that sklearn's implementation is substantially faster. This is because
  our implementation is based on the $O(n^2d)$ decision stump learning algorithm and sklearn's implementation presumably uses the faster $O(nd\log n)$
  decision stump learning algorithm that we discussed in lecture.
  \centerfig{0.7}{./figs/q6_5_decisionBoundary}
  \begin{answer}
    From the decision error graph we can see that the error rate algorithm stops to improve accuracy after 4 depths of tree, while the others go to almost 0 errors. The reason behind this is that it's hard to gain accuracy directly through each split. With the error rate decision rule, a split needs to change the majority of y in order to update the prediction, but it can be updated with marginal change of entropy with the informaiton gain rule.
  \end{answer}

  \subsection{Comparing implementations \pts{2}}

  In the previous section you compared different implementations of a machine learning algorithm. Let's say that two
  approaches produce the exact same curve of classification error rate vs. tree depth. Does this conclusively demonstrate
  that the two implementations are the same? If so, why? If not, what other experiment might you perform to build confidence
  that the implementations are probably equivalent?

  %
  \begin{answer}
    In general, we are not sure whether they are the same. Firstly, they might have different errors but same number of errors. Secondly, it might just be a coincidence with this exact sample. We can do experiments with different samples and directly compare their classification result.
  \end{answer}


  \clearpage \section{Appendix}
  Code for \ref{sec6.2}
  \label{code6.2}\begin{minted}{python}
    class DecisionStumpErrorRate:
      y_hat_yes = None
      y_hat_no = None
      j_best = None
      t_best = None
  
      def fit(self, X, y):
          n, d = X.shape
  
          # Get an array with the number of 0's, number of 1's, etc.
          count = np.bincount(y)
  
          # Get the index of the largest value in count.
          # Thus, y_mode is the mode (most popular value) of y
          y_mode = np.argmax(count)
  
          self.y_hat_yes = y_mode
          self.y_hat_no = None
          self.j_best = None
          self.t_best = None
  
          # If all the labels are the same, no need to split further
          if np.unique(y).size <= 1:
              return
  
          minError = np.sum(y != y_mode)
  
          # Loop over features looking for the best split
          for j in range(d):
              for i in range(n):
                  # Choose value to equate to
                  t = X[i, j]
  
                  # Find most likely class for each split
                  weakly_less = X[:, j] <= t
                  y_yes_mode = utils.mode(y[weakly_less])
                  y_no_mode = utils.mode(y[~weakly_less])  # ~ is "logical not"
  
                  # Make predictions
                  y_pred = y_yes_mode * np.ones(n)
                  y_pred[np.round(X[:, j]) > t] = y_no_mode
  
                  # Compute error
                  errors = np.sum(y_pred != y)
  
                  # Compare to minimum error so far
                  if errors < minError:
                      # This is the lowest error, store this value
                      minError = errors
                      self.j_best = j
                      self.t_best = t
                      self.y_hat_yes = y_yes_mode
                      self.y_hat_no = y_no_mode
  
  
      def predict(self, X):
          n, d = X.shape
          if self.j_best is None:
              return self.y_hat_yes * np.ones(n)
  
          y_hat = np.zeros(n)
  
          for i in range(n):
              if X[i, self.j_best] <= self.t_best:
                  y_hat[i] = self.y_hat_yes
              else:
                  y_hat[i] = self.y_hat_no
  
          return y_hat
    \end{minted}

Back to section \ref{sec6.2}.

\clearpage
Code for \ref{sec6.3}
\label{code6.3}\begin{minted}{python} 
  class DecisionStumpInfoGain:
    # This is not required, but one way to simplify the code is
    # to have this class inherit from DecisionStumpErrorRate.
    # Which methods (init, fit, predict) do you need to overwrite?
    y_hat_yes = None
    y_hat_no = None
    j_best = None
    t_best = None

    def fit(self, X, y):
        n, d = X.shape

        # Get an array with the number of 0's, number of 1's, etc.
        count = np.bincount(y)

        # Get the index of the largest value in count.
        # Thus, y_mode is the mode (most popular value) of y
        y_mode = np.argmax(count)

        self.y_hat_yes = y_mode
        self.y_hat_no = None
        self.j_best = None
        self.t_best = None
        p = count/y.size

        # If all the labels are the same, no need to split further
        if np.unique(y).size <= 1:
            return

        minEntropy = entropy(p)

        # Loop over features looking for the best split
        for j in range(d):
            for i in range(n):
                # Choose value to equate to
                t = X[i, j]

                # Find most likely class for each split
                weakly_less = X[:, j] <= t
                y_yes_mode = utils.mode(y[weakly_less])
                y_no_mode = utils.mode(y[~weakly_less])
                y_yes = y[weakly_less]
                y_no = y[~weakly_less] 
                p_yes = np.bincount(y_yes)/y_yes.size
                p_no = np.bincount(y_no)/y_no.size

                NewEntropy = y_yes.size/y.size * entropy(p_yes) + y_no.size/y.size *entropy(p_no)

                # Compare to minimum error so far
                if NewEntropy < minEntropy:
                    # This is the lowest error, store this value
                    minEntropy = NewEntropy
                    self.j_best = j
                    self.t_best = t
                    self.y_hat_yes = y_yes_mode
                    self.y_hat_no = y_no_mode

    def predict(self, X):
        n, d = X.shape

        if self.j_best is None:
            return self.y_hat_yes * np.ones(n)

        y_hat = np.zeros(n)

        for i in range(n):
            if X[i, self.j_best] <= self.t_best:
                y_hat[i] = self.y_hat_yes
            else:
                y_hat[i] = self.y_hat_no

        return y_hat
  \end{minted}
Back to section \ref{sec6.3}.
\end{document}
